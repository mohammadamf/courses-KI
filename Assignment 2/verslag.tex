
%
% verslag.tex   18.2.2019
% Voorbeeld LaTeX-file voor verslagen bij Kunstmatige Intelligentie
% http://www.liacs.leidenuniv.nl/~kosterswa/AI/verslag.tex
%
% Gebruik:
%   pdflatex verslag.tex
%

\documentclass[10pt]{article}

\parindent=0pt

\usepackage{fullpage}

\frenchspacing

\usepackage{microtype}

\usepackage[english,dutch]{babel}

\usepackage{graphicx}
%\usepackage{hyperref}

\usepackage{listings}
\usepackage{subcaption}

% Er zijn talloze parameters ...
\lstset{language=C++, showstringspaces=false, basicstyle=\small,
  numbers=left, numberstyle=\tiny, numberfirstline=false, breaklines=true,
  stepnumber=1, tabsize=8,
  commentstyle=\ttfamily, identifierstyle=\ttfamily,
  stringstyle=\itshape}

\usepackage[setpagesize=false,colorlinks=true,linkcolor=red,urlcolor=blue,pdftitle={Het grote probleem},pdfauthor={David Kleingeld}]{hyperref}

\author{David Kleingeld, s1432982}
\title{RoboSquares, color coordinated multi agent square formation}

\begin{document}

\selectlanguage{dutch}

\maketitle

\section{Inleiding}

Robocom is een computer "spel" waarbij de spelers bord stukken ook wel robots programmeren. Wanneer de programmas af zijn start het spel ook wel de simulatie. De robots voeren dan hun acties uit. Traditioneel is het doel om de robots van de tegenstander uit te schakelen \cite{robocom_program}. Hier proberen we de robots echter juist samen te laten werken. Specifieker nog we proberen ze een vierkant te laten vormen. Een vierkant met een andere kleur in het midden dan aan de rand.

\section{Theorie}

De robots die we gebruiken hebben slechts 12 verschillende operaties \cite{robocom_docs}. Ze kunnen bewegen, kijken wat er voor hun staat, een nieuwe robot creëren, het programma van een andere robot veranderen en zichzelf verwijderen. Daarnaast bied de assembler achtige taal waarin de robots geschreven worden: variabelen in de vorm van registers, add en subtract operaties op die registers en basale logica via een compare operatie en jumps. Operaties kosten tijd, ook wel tiks. Sommige operaties kosten echter meer tiks dan andere. Voor elke robot word na elkaar 1 tick gesimuleerd. Het maken van nieuwe robots is het duurst gevolgd door het overzetten van programmas.

Wanneer een robot een nieuwe robot maakt noemen we deze ouder en kind. Het kind krijgt dezelfde kleur als de ouder en start met een leeg programma. Een nieuwe robot staat standaard uit en kan aangezet worden door een robot ernaast. 

Alle robots starten met hetzelfde programma op een willekeurige positie in de spelwereld. Deze is een vierkant waarbij de randen aan elkaar gelijmd zijn, als je aan een kant over de rand beweegt kom je bij de tegenoverstaande rand de wereld weer in. De afstand tussen de randen is beperkt.

Initieel proberen we met de bovenstaande beperkingen de verschillende agenten (robots) een 3x3 vierkant te vormen. We kijken ook wat er gebeurt met de agenten als we er initieel drie plaatsen. Vormt er een 5x5 vierkant? 

De bovenstaande uiteenzetting is samen te vatten de in AI veel gebruikte 4 PEAS eigenschappen om de agent te beschrijven: 

\begin{enumerate}
    \item Performance: vinden van de andere kleur, de ander kleur volledig omsluiten
    \item Environment: k bij k discrete wereld, tegenoverstaande randen aan elkaar, begint op een willekeurige plek
    \item Actuatoren: roteren, verplaatsen, nieuw robots maken, robots activeren, robots deactiveren, robots hun programmering overschrijven.
    \item bekijken wat er voor zich staat: zelfde kleur robot, andere kleur robot, niks
\end{enumerate}

\begin{description}
 \item[deterministisch] Nee, de startpositie is onbepaald en de volgorde waarin de robots acties gesimuleerd worden is onbepaald. 
 \item[statisch] Ja, in het geval van de uitdaging om een vierkant te maken is er geen verandering aan de wereld terwijl onze agenten nadenken. Echter de robocom wereld is zeker niet statisch wanneer er andere niet door ons geschreven agenten in geplaatst worden.
 \item[obeserveerbaar] slechts deels, voor een enkele agent is alleen zichtbaar wat er op dat moment recht voor de agent staat.

 \item[multi-agent] Ja, we werken immers vanaf het begin met meer dan 1 agent, wel coöperatief.
 \item[discreet] Elke actie kost een vast aantal tiks, alle agenten worden elke tik een stap verder gesimuleerd. Dus volledig discreet.
 \item[episodisch] Nee, wanneer het vierkant gevormd is zijn we klaar.
\end{description}


\section{Methode}

Om een vierkant te kunnen krijgen bestaand uit twee kleuren starten we met twee robots. Hoe bepalen we welke kleur de binnenkant en buitenkant worden? We hebben geen instructie met een willekeurige uitkomst nog kunnen we van te voren bepalen welke robot het midden word. Er zijn echter wel willekeurige elementen in het spel. De startpositie en oriëntatie is elke keer anders en

 de volgorde waarin de tiks gesimuleerd worden is willekeurig. We gebruiken dit laatste. Als twee robots elkaar uit proberen te schakelen op hetzelfde tijdstip in de simulatie slaagt degene waarvan de laatste tick van de uitschakel operatie als eerst gesimuleerd word. Dit gebruiken we om te bepalen welke kleur het midden van het vierkant word.

\label{lijnen}
Met een verdeling tussen binnenkant en buitenkant kunnen we een procedure schrijven die vanuit een vaste positie altijd een vierkant vormt. Echter we hebben geen vaste start positie maar een willekeurige, hoe krijgen we de verschillende kleuren robots naar een vaste positie zodat we een vierkant kunnen vormen? En een vervolg vraag, wat is een positie? Immer we spelen op een rand loos bord zonder herkenningspunten. We definiëren de positie als relatief tot de andere kleur. Zo gedefinieerd krijgen we de vraag: hoe zorgen we dat de verschillend gekleurde robots elkaar binnen een beperkte tijd vinden? De robots over het bord bewegen werkt niet altijd, richt ze parallel en ze zullen elkaar nooit vinden hoe ingewikkeld zo ook bewegen. Laat de robots een rechte lijn maken door telkens een kloon voor zich te plaatsen en ze vinden elkaar wel indien de start oriëntatie orthogonaal was zoals in \ref{fig:0a}. 

\begin{figure}[h!]
  \centering
  \includegraphics[width=0.3\linewidth]{orthogonaal.png}
  \caption{Kleuren vinden elkaar na een orthogonale start.}
  \label{fig:0a}
\end{figure}

Dit werkt niet wanneer de start oriëntatie parallel is (zie \ref{fig:0b}). Dit lossen we op door de robots nog een lijn haaks op de al bestaande te laten maken. Die lijn is namelijk gegarandeerd orthogonaal op de eerste lijn van de andere kleur. 

\begin{figure}[h!]
  \centering
  \includegraphics[width=0.3\linewidth]{parallel.png}
  \caption{Kleuren vinden zichzelf na een parallelle start, daarna draaien ze voordat ze weer verder een lijn maken.}
  \label{fig:0b}
\end{figure}

Wanneer de lijn de robots van een andere kleur bereikt heeft verwijderen we de hele lijn tot naast de andere kleur. De andere kleur krijgt de opdracht hun lijnen ook te verwijderen. We definieren verder twee richtingen in de lijnen, upstream en downstream. Upstream is richting de eerste robot/eerder geplaatste robots, downstream tegenovergesteld.

\section{Uitvoering}

Hier bespreken we de opzet van het programma, eerst in de volgorde waarmee het bord evolueert daarna benoemen we nog wat implementatie specifieke problemen en oplossingen.

\subsection{vinden}
We beginnen met het laatste probleem uit de methode, hoe de robots elkaar vinden. Elke nieuw gecreerde robot, elke nieuw stuk van lijnen \ref{lijnen} begint bij de code in bank 1:

\begin{verbatim}
  Scan #1           ;kijk op er een robot voor ons staat, sla op in Reg 1
  comp #1, 0        ;is het veld voor ons leeg?
  bJump 3, 1        ;nee => kijk wat voor bot het is
  bJump 2, 1        ;ja => maak een bot voor ons
\end{verbatim}

We zien hier dat er of verder word gewerkt aan de rechte lijn of gekeken word welke kleur voor de robot staat. Immers kan het zijn dat de lijn terug bij het begin is of dat de andere kleur gevonden is. Met de andere kleur gevonden word de robot die als eerst de ander deactiveert (ook wel master) de buitenkant van het vierkant. De nu gedeactiveerde robot noemen we verder de slave.

\subsection{opruimen}
Het bord ligt nu echter nog vol met lijnen deze ruimen we eerst op: de slave en master draaien een rondje, indien er tijdens het rondje een robot van dezelfde kleur voor ze staat geven ze die het "Destroy\_upstream" programma:

\begin{verbatim}
Bank Destroy_upstream ;bank 5
  Turn 0              ;nee -> keer om
  Turn 0              ;ja -> draai 90 naar rechts
  Trans 5, 1
  Die
\end{verbatim}

Dit laat elke robot zich omdaaien, dit programma programma in bank 1 van de robot tegenover zich zetten en dan zichzelf verwijderen. Zo blijft uiteindelijk alleen voorste node over. Een variant hierop word voor het rondje gestart als er een bot met dezelfde kleur voor de robot staat. Bij deze variant word niet rondgedraaid. Dit zorgt dat een slave in het midden van een lijn geïsoleerd word zodat deze niet later door het Destroy\_upstream programma opgeruimd word.

\subsection{vierkant maken}
Nu rest slechts het af maken van het vierkant. Op dit punt zijn er altijd minstens 1 master-slave pair over waarbij de master 90 graden van de slave weg gedraaid georiënteerd is. De master maakt een simpele bouw robot en geeft die het make\_circle programma. Daarna draait de master zich om en activeert deze alles wat langs komt.


\begin{verbatim}
Bank make_circle ;bank 8
  ;look whats right
  turn 1
  scan #1
  
  ;other color right?
  comp #1, 1 
  jump @build_left  ;nee -> build 
  jump @build_front ;ja -> turn back front then build 

  @build_left
  Create 2, 1, 1
  Trans 1, 1        ;bank naar de nieuwe robot
  Set \%Active, 1
  Set #Active, 0
  
  @build_front
  turn 0            ;turn us back front again
  Create 2, 1, 1
  Trans 1, 1        ;bank naar de nieuwe robot
  Set \%Active, 1
  Set #Active, 0
\end{verbatim}

Het make\_circle programma werkt als volgt: indien er rechts niet de slave robot ligt zijn we op een hoekpunt van het vierkant dan moeten we naar binnen draaien voor we weer een bouw robot maken. Anders bouwen we een nieuwe bouw robot recht voor ons. Wanneer we weer naast de master staan is het vierkant compleet, voordat we nog een bouw robot kunnen maken worden we door de master ge-deactiveerd. Nu zijn alle robots klaar en staat er minstens 1 vierkant.

\subsection{details}
De beschrijving hierboven slaat wat kleine maar belangrijke details over. Hier staan we daar kort bij stil. Na het bouwen van het volgende stukje van de lijn blijft de robot actief op de 5e regel van de eerste bank. Bij het weer verwijderen van de lijn kopiëren we de verwijder functie vanuit een andere robot naar bank 1. Deze begint daardoor te draaien vanaf regel 5. Daarom start de verwijder functie code pas op regel 5.

Bij het upstream verwijderen draait de robot zich telkens 180 graden om alvorens het programma verder te kopiëren en zichzelf te verwijderen. Echter indien we bij vanuit de orthogonale 2e gemaakte lijn de 1e lijn moeten gaan verwijderen liggen de oude robots niet achter ons maar links en rechts. Zie robot Z, die naar boven georiënteerd staat hieronder:

\begin{verbatim}
   Y
   Y
 XXZXX
\end{verbatim}

Dit lossen we op door voor het draaien te kijken of dit onderdeel van de lijn gedraaid is. Bij het aanleggen van een lijn zetten we een register op 1 wanneer gedraaid word. Indien gedraaid is word geen 180 graden gedraaid maar 90 tegen de originele draairichting in.

Het opruimen van alle lijnen kost tijd, wanneer we direct nadat de master klaar is met Destroy\_Same\_Color het vierkant maken kan het vierkant mee verwijderd worden. Dus wachten we eerst met een simpele while loop: 

\begin{verbatim}
  Set #10, 300
  @while_loop
  sub #10, 1
  comp #10, 0                       ;klaar met loop?
  Jump @while_loop                  ;nee -> terug naar boven
\end{verbatim}

\pagebreak

\section{Resultaten} 

In meer dan 90\% van de gevallen word een of meer vierkanten bereikt wanneer twee robots geplaatst worden. We krijgen twee vierkanten, zie figuur \ref{fig:1} voor hoe deze ontstaan. In 10\% van de andere gevallen blijft maar 1 robot over.  

\begin{figure}[h!]
  \centering
  \begin{subfigure}[b]{0.4\linewidth}
    \includegraphics[width=\linewidth]{word2.png}
    \caption{De twee kleuren vinden elkaar op 2 locaties}
  \end{subfigure}
  \begin{subfigure}[b]{0.4\linewidth}
    \includegraphics[width=\linewidth]{2.png}
    \caption{Het resultaat van links}
  \end{subfigure}
  \caption{Twee twee kleurige vierkanten.}
  \label{fig:1}
\end{figure}

Wanneer we starten met drie robots kunnen de resultaten globaal in 4 catogorieen ingedeeld worden:

\begin{description}
 \item[Time out] Er vormt een vierkant van 2 kleuren, de overige kleur blijft draaien tot deze het bestaande vierkant vind maar dit lukt niet op tijd \ref{fig:drie1}.
 \item[Gemengd] Er ontstaat een mislukt vierkant uit meerdere kleuren omdat meerdere robots master worden of een vierkant nog niet af was terwijl de volgende lijn raakte.\ref{fig:drie3} \ref{fig:drie_fout}.
 \item[geluk] Er ontstaat een vierkant van 2 kleuren en dit word daarna pas geraakt door een lijn van de derde kleur wanneer het helemaal af is \ref{fig:drie_geluk}.
\end{description}

\begin{figure}[h!]
  \centering
  \begin{subfigure}[b]{0.4\linewidth}
    \includegraphics[width=\linewidth]{drie1.png}
    \caption{Time out} \label{fig:drie1}
  \end{subfigure}
  \begin{subfigure}[b]{0.4\linewidth}
    \includegraphics[width=\linewidth]{drie3.png}
    \caption{Gemengd, meerdere masters} \label{fig:drie3}
  \end{subfigure}
  \begin{subfigure}[b]{0.4\linewidth}
    \includegraphics[width=\linewidth]{drie_fout.png}
    \caption{Gemengt, 3e lijn op niet af vierkant} \label{fig:drie_fout}
  \end{subfigure}
    \begin{subfigure}[b]{0.4\linewidth}
    \includegraphics[width=\linewidth]{drie_geluk.png}
    \caption{Een 5x5 vierkant} \label{fig:drie_geluk}
  \end{subfigure}
  \caption{Vier pogingen om een 3 kleurig vierkant te krijgen.}
  \label{fig:coffee}
\end{figure}

\section{Conclusie}
Wanneer er geen vierkant ontstaat is dit terug te brengen tot een startsituatie waardoor de robots net verkeerd elkaar vinden. Door deze te bestuderen kan het programma subtiel aangepast worden om hier rekening mee te houden. Het nadeel is dat dit veel tijd kost, niet alleen om te maken maar ook de run-time van het programma neemt toe.

Bij het maken van een 3 kleurig vierkant gaat veel mis. De time-outs kunnen we voorkomen door de agent te programmeren na 2 bochten over te gaan op een minder fijnmazig zoek patroon immers is er dan al een 3x3 vierkant gezien de lijnen weg zijn. 

Door de robots die de lijnen vormen informatie door te laten geven kan een globale staat gemaakt worden waarmee bepaald kan worden of alle kleuren in contact zijn en waar. Zo kunnen dubbele vierkanten voorkomen worden en beter gecoördineerd worden voor het 3 kleurige vierkant. Dit word waarschijnlijk zeer complex.

Om meer complex gedrag te kunnen implementeren is een hoger niveau van abstractie nodig. We kunnen functies implementeren door registers aan te wijzen als return registers en als registers voor argumenten. Dit maakt code reuse mogelijk en complexer gedrag.

\bibliography{verslag.bib}
\bibliographystyle{IEEEtran}

\pagebreak
\section*{Appendix: Code}

\smallskip

\begin{verbatim}; RoboCom program

Published  Name        KIASS2                   ; Name of this program
Published  Author      YOUR_NAME                ; Name of author
Published  EMail       YOUR_EMAIL_ADDRESS       ; Author's e-mail address
Published  Country     YOUR_HOME_COUNTRY_HERE   ; Author's home country
Published  Comment     MAKE_A_COMMENT           ; A comment on this prog
Secret     Password    YOUR_PASSWORD_HERE       ; Password for competitons

Published  OpenSource  yes                      ; This prog is open source
Published  Language    RC300                    ; Written in RC300 language
Published  OptionSet   RC3 Standards            ; Recommended OptionSet

;reg 1: stores scan result
;reg 5: did we turn left
;reg 20: are we master?
;reg 10: loop counter

Bank Main           ;bank 1
  Scan #1           ;kijk op er een robot voor ons staat, sla op in reg 1
  comp #1, 0        ;is het veld voor ons leeg?
  bJump 3, 1        ;nee => kijk wat voor bot het is
  bJump 2, 1        ;ja => maak een bot voor ons
  @standby
  Jump @standby
  
Bank No_Bot         ;bank 2
  Create 2, 8, 0    ;nieuwe robot, instructie set 2, 5 banken, kan niet bewegen (0) 
  
  ;kan mislukken door race conditions daarom:
  Scan #1             ;kijk op er een robot voor ons staat, sla op in reg 1
  comp #1, 1          ;is er nu een bot met een andere kleur?  
  Jump   @trans_regs  ;nee => ga verder
  bJump 4, 1          ;ja => race condition, beslis master vs slave
  
  @trans_regs
  Trans 1, 1        ;bank naar de nieuwe robot
  Trans 2, 2        ;bank naar de nieuwe robot
  Trans 3, 3        ;bank naar de nieuwe robot
  Trans 4, 4        ;bank naar de nieuwe robot
  Trans 5, 5        ;bank naar de nieuwe robot
  Trans 6, 6        ;bank naar de nieuwe robot
  Trans 7, 7        ;bank naar de nieuwe robot
  Trans 8, 8        ;bank naar de nieuwe robot
  Set %Active, 1    ;activate the remote bot
  bJump 1, @standby ;go standby
  
Bank Bot_Detected   ;bank 3
  comp #1, 2        ;is de bot dezelfde kleur?
  bJump 4, 1        ;nee => beslis master/slave
  turn 0            ;ja => draai linksom
  set #5, 1         ;onthoud dat we naar links gedraaid hebben
  bJump 2, 1        ;plaats een robot voor ons
  
Bank Found_Other    ;band 4
                    ;only continue if master
  Set %Active, 0    ;deactivate the remote bot to become master
                    ;only reach this part if master
  set #20, 1        ;mark this robot as center
  
  ;tell the other color on in front to destroy all same colord save itself
  Trans 7, 1        ;transfer destroy same color program
  Set %Active, 1    ;activate the slave bot
  ;bJump 1, @standby ;FIXME remove this (testing only)
  
  ;delete the bots we used to get here en maak de circle
  bJump 7, 1
     
  
  ;TODO go around
  bJump 1, @standby ;TODO placeholder

Bank Destroy_upstream ;bank 5
  Jump 5              ;make sure then real code start at 5, since standby runs at 5
  Jump 4              ;make sure then real code start at 5, since standby runs at 5
  Jump 3              ;make sure then real code start at 5, since standby runs at 5
  Jump 2              ;make sure then real code start at 5, since standby runs at 5
  Jump 1              ;make sure then real code start at 5, since standby runs at 5
  Comp #5, 1          ;zijn we naar links gedraaid?
  Turn 0              ;nee -> keer om
  Turn 0              ;ja -> draai 90 naar rechts
  Trans 5, 1
  Die

Bank Destroy_downstream ;bank 6
  Jump 5                ;make sure then real code start at 5, since standby runs at 5
  Jump 4                ;make sure then real code start at 5, since standby runs at 5
  Jump 3                ;make sure then real code start at 5, since standby runs at 5
  Jump 2                ;make sure then real code start at 5, since standby runs at 5
  Jump 1                ;make sure then real code start at 5, since standby runs at 5
  Trans 6, 1
  Die
  
Bank Destroy_Same_Color ;bank 7
  Jump @scan1           ;make sure then real code start at 5, since standby runs at 5
  Jump @scan1           ;make sure then real code start at 5, since standby runs at 5
  Jump @scan1           ;make sure then real code start at 5, since standby runs at 5
  Jump @scan1           ;make sure then real code start at 5, since standby runs at 5
  Jump @scan1           ;make sure then real code start at 5, since standby runs at 5
  
  @scan1
  Scan #1
  comp #1, 2        ;bot met zelfde kleur?
  Jump @scan2       ;nee -> draai en scan nog 3 maal
  Trans 6, 1        ;ja -> transfer destroy downstream program ;leaving out can do intresting things....
  
  @scan2
  turn 1
  Scan #1
  comp #1, 2        ;bot met zelfde kleur?
  Jump @scan3       ;nee -> draai en scan nog 3 maal
  Trans 5, 1        ;ja -> transfer destroy upstream program 

  @scan3
  turn 1
  Scan #1
  comp #1, 2        ;bot met zelfde kleur?
  Jump @scan4       ;nee -> draai en scan nog 3 maal
  Trans 5, 1        ;ja -> transfer destroy upstream program 

  @scan4
  turn 1
  Scan #1
  comp #1, 2                        ;bot met zelfde kleur?
  Jump @done_turning                ;nee -> doe niks meer
  Trans 5, 1                        ;ja -> transfer destroy upstream program 
  @done_turning
  
  comp #20, 0                       ;bot slave?
  bJump 8, @start_building          ;nee -> maak circle
  comp #10, 0                       ;ja -> first loop?
  Set #Active, 0                    ;nee -> doe niks meer
  
  Set #10, 1000
  @while_loop2
  sub #10, 1
  comp #10, 0                       ;klaar met loop?
  Jump @while_loop2                  ;nee -> terug naar boven
  
  Set #10, 1                        ;ja -> wacht en doe dit nog een keer
  jump @scan1                       ;ja -> wacht en doe dit nog een keer
  

Bank make_circle ;bank 8
  ;look whats right
  turn 1
  scan #1
  
  ;other color right?
  comp #1, 1 
  jump @build_left  ;nee -> build 
  jump @build_front ;ja -> turn back front then build 

  @build_left
  Create 2, 1, 1
  Trans 1, 1        ;bank naar de nieuwe robot
  Set %Active, 1
  Set #Active, 0
  
  @build_front
  turn 0            ;turn us back front again
  Create 2, 1, 1
  Trans 1, 1        ;bank naar de nieuwe robot
  Set %Active, 1
  Set #Active, 0
  
  @start_building
  ;wait for all robots around us to get destroyed
  Set #10, 3000
  @while_loop
  sub #10, 1
  comp #10, 0                       ;klaar met loop?
  Jump @while_loop                  ;nee -> terug naar boven

  ;maak de bot die de circle af gaat maken
  Create 2, 1, 1
  Trans 8, 1        ;bank naar de nieuwe robot
  Set %Active, 1                    ;
  
  ;zodra de circle compleet is stop het bouwen
  turn 0
  turn 0
  @loop
  Set %Active, 0
  jump @loop
\end{verbatim}


\end{document}
