
%
% verslag.tex   18.2.2019
% Voorbeeld LaTeX-file voor verslagen bij Kunstmatige Intelligentie
% http://www.liacs.leidenuniv.nl/~kosterswa/AI/verslag.tex
%
% Gebruik:
%   pdflatex verslag.tex
%

\documentclass[10pt]{article}

\parindent=0pt

\usepackage{fullpage}

\frenchspacing

\usepackage{microtype}

\usepackage[english,dutch]{babel}

\usepackage{graphicx}
%\usepackage{hyperref}

\usepackage{listings}
% Er zijn talloze parameters ...
\lstset{language=C++, showstringspaces=false, basicstyle=\small,
  numbers=left, numberstyle=\tiny, numberfirstline=false, breaklines=true,
  stepnumber=1, tabsize=8,
  commentstyle=\ttfamily, identifierstyle=\ttfamily,
  stringstyle=\itshape}

\usepackage[setpagesize=false,colorlinks=true,linkcolor=red,urlcolor=blue,pdftitle={Het grote probleem},pdfauthor={Victor Erslag}]{hyperref}

\author{David Kleingeld}
\title{Verslag KI opdracht schaken }

\begin{document}

\selectlanguage{dutch}

\maketitle

\section{Inleiding}

Dit is de eerste programmeeropdracht voor het college KI \footnote{zie ook wel: \url{http://liacs.leidenuniv.nl/~kosterswa/AI/}}. Hier schrijven we een simpel programma dat het laatste stuk van het spel schaken zo goed mogelijk probeert te spelen. We nemen aan dat er aan het eind van het spel $1$ witte dame, $1$ witte koning en $1$ zwarte koning over zijn. Dit kan op 3 manieren eindigen: schaakmat, pat stelling \footnote{ookwel stalemate in het engels} of het spel gaat eeuwig door. Dit laatste gebeurt als de witte koningin gevangen is. Er zijn strategieen om dit zo snel mogelijk te doen \footnote{zie ook wel: \url{https://www.schaakzone.nl/eindspel/matzetten-toren-koning.php}}. Hier proberen we zonder specifieke voorkennis methodisch en systematisch dit probleem op te lossen.

\section{Theorie}

We willen hier dus een rationele agent schrijven. De meest voor de hand liggende oplossing voor een rationele agent is om alle mogelijkheden af te wegen en uit te rekenen. Alle mogelijke zetten voor ons schaakprogramma is echter onmogelijk. Een redelijke schatting is $\approx$ 30 \cite{Shannon:1988} mogelijke zetten en een partij lengte van $\approx$ zetten \cite{kosters:lecture2:sheet9}. Dit geeft ongeveer $30^{100} \approx 5x10^{147}$ regels voor de tabel. Voor het complete schaakspel word dit alleen maar erger. Vanuit de beginpositie zijn er alleen al $10^{120}$ variaties \cite{Shannon:1988}.

Shannon stelt een chess program voor dat posities op het bord een score geeft en daar mee de zetten selecteerd. Hier proberen we iets soort gelijks. Schaak is een volledig observeerbaar, deterministisch, sequenteel en discreet probleem. Immers het hele bord kan overzien worden, en komen geen kansen voor, de spelers zijn om de beurt aan zet en er zijn een gelimiteerd aantal zetten mogelijk. Dit betekend dat een doel gebaseerde agent, die mogelijke acties aan een bestaand model weegt, een goede aanpak kan zijn. Zoals Shannon voorstelt.

\section{Methode}

We schrijven 2 agenten. Als eerst schrijven doel gerichte agent die alle mogelijke zetten een score geeft en die met de hoogste score uitvoert. Zetten waarbij de koningin naast de zwarte koning komt te staan vermijden we, immers dan raken we de koningin snel kwijt, door ze een score van 1 te geven. Voor alle andere zetten berekenen we de score als volgt:

\begin{equation}
  S = 20*\frac{1}{1+N_b}
      + 10*\frac{1}{1+D_q}
      + 10*\frac{1}{1+D_k}
      + 2*N_{w queen}
      + 2*N_{w king}
\end{equation}

Waar $N_b$: het aantal mogelijke zetten voor de zwarte koning, $D_q$: de afstand van de witte dame tot de zwarte koning, $D_k$:, de afstand van de witte koning tot de zwate, $N_{w queen}$: number of moves for the white queen, $N_{w king}$: number of moves for the white king.

Als de zwarte koning schaak staat voegen we 15 toe aan de score.

Als tweede schrijven we een agent die gebruik maakt van de monte carlo simulatie. Hiervoor gaan we weer alle mogelijke zetten langs. We doen elke mogelijke zet en simuleren het verloop van het spel 1000 keer waarbij beide partijen na de zet random vervolgzetten doen zijn. De eindsituatie van al deze simulaties krijgt dan een score. Deze word anders bepaald dan hierboven namelijk:

\begin{equation}
  S = 1000*U + \Sigma_{i=0}^{1000} (N_{i})
\end{equation}
Hier is $N_i$ het aantal zetten dat de gesimuleerde partij korter was dan het maximum. U is het aantal partij-en waarbij de witte speler won.

\section{Implementatie}
Beide agenten zijn geimplementeerd in C++ als uitbreiding op de gegeven \href{http://liacs.leidenuniv.nl/~kosterswa/AI/chess.cc}{\underline{skeletcode}}. Er zijn ook een aantal kleine wijzigingen gemaakt om het verkrijgen van statistiek te vergemakkelijken. Zo was er in de orginele skeletcode een visuele weergave van het speelbord. Deze is nu compile time uit te zetten zodat deze niet telkens gecopieerd word in de monte carlo simulatie.

\bigskip



\section{Experimenten}

We houden voor de twee geimplementeerde agenten en een agent die alleen willekeurig speelt per partij bij:
\begin{enumerate}
\item De uitkomst
\item Het aantal beurten
\item Voor elke beurt de score
\item de afstand tussen de witte en zwarte stukken
\end{enumerate}

De uitkomsten met een maximale partijlengte van 200 zetten zijn:
\begin{center}
\begin{tabular}{l|l|l|l}
 uitkomst & willekeurig agent (\%)& simpele agent (\%) & monte carlo agent(\%)\\
\hline
wins &1.3\% & 21\% & 88\%\\
stalemates &7.6\% & 36\% & 12.5\%\\
simple ties &91\% & 0\% & 0\%\\
stopped &0\% &42\% & 0\%\\
\end{tabular}
\end{center}

\begin{center}
\begin{tabular}{l|l|l|l}
 uitkomst & willekeurig agent& simpele agent & monte carlo agent\\
\hline
wins & 55 & 429 & 350\\
stalemates & 304 & 722 & 849\\
simple ties & 3640 & 0 & 0\\
stopped &1 &849 & 0\\
\end{tabular}
\end{center}

Het gebrek aan simple ties bij de monte carlo agent is simpel te verklaren. Omdat gemiddeld gezien het verplaatsen van de dame naar een veld naast de zwarte koning een goede zet is komt dit vaak uit de monte carlo simulaties. Echter deze zet is niet realistish. Daarom word deze er uit gefilterd. Indien de dame nooit naast de zwarte koning staat kan deze ook nooit gevangen worden, er zijn daarom geen simpel ties. Dezelfde regel bestaat voor de simpele agent, daarom daar ook geen simpel ties.

Hieronder zien we histogrammen van het aantal zetten per partij voor de 4 verschillende uitkomsten:

\begin{figure}[!htbp]
\begin{center}
\includegraphics[scale=0.6]{willikeurige_agentn_moves}
\end{center}
\caption{Voor de Willikeurige Agent: Histogram van het aantal zetten nodig om een van de verschillende eindstaten te bereiken. De eind staat stopped is weg gelaten aangezien deze altijd na 200 zetten bereikt word}
\end{figure}

\begin{figure}[!htbp]
\begin{center}
\includegraphics[scale=0.6]{simpele_agentn_moves}
\end{center}
\caption{Voor de Simpele agent: histogram van het aantal zetten nodig om een van de verschillende eindstaten te behalen. De eind staat stopped is weg gelaten aangezien deze altijd na 200 zetten bereikt word}
\label{hist_simple}
\end{figure}

\begin{figure}[!htbp]
\begin{center}
\includegraphics[scale=0.6]{monte_carlo_agentn_moves}
\end{center}
\caption{Voor de Monte Carlo agent: histogram van het aantal zetten nodig om een van de verschillende eindstaten te behalen. De eind staat stopped is weg gelaten aangezien deze altijd na 200 zetten bereikt word}
\end{figure}

\section{Conclusie}

Met wat geluk of genoeg tijd en een willekeurig spelende tegenstander kan een doel gebaseerde agent beter presteren dan een willikeurige agent. Echter voor een groot deel is deze simpele agent nog afhankelijk van de zetten van de tegenstander. Er word niet gekeken naar de toekomstige situatie. Dit is gelijk te zien aan het histogram van het aantal zetten (zie figuur: \ref{hist_simple}). We zien dat het aantal beurten over het algemeen nog langer duurt dan bij de willekeurige agent. Een goede vervolg stap is het verder ontmoedigen van stalemates in de score berekening.

Wat gelijk opvalt bij de Monte Carlo agent is het maximum aantal zetten. Dit is onder de 30. Ook zien we dat deze agent heel vaak binnen enkele zetten al klaar is of binnen enkele zetten een kritisch fout maakt. De sterke piek in stalemates in de eerste paar zetten doet vermoedde dat een aanpassing aan de score voor elke mogelijke zet zou kunnen helpen.


\bibliography{verslag.bib}
\bibliographystyle{ieeetr}

\section*{Appendix: Code}

Er is gebruik gemaakt van de \href{http://www.liacs.leidenuniv.nl/~kosterswa/AI/iets.cc}{\underline{skeletcode}} die te vinden is via
de website van het college.
De code van het programma is als volgt:

\smallskip

\lstinputlisting{chess.cc}

\end{document}